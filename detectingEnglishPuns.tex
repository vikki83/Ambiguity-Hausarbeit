\documentclass[10pt]{article}
%\documentstyle{makeidx}
\usepackage{imakeidx}
\makeindex
\usepackage{latexsym,amssymb}
\usepackage[
	%leqno,
	leqno,
	%fleqn,
	]{amsmath}
\usepackage{linguex}
\makeatletter
	\let\c@ExNo\c@equation
\makeatother
\usepackage{xcolor}
\newcommand{\Timm}[1]{{\color{blue}#1}}
\usepackage{soul}   % for text highlighting
\setstcolor{red} 	
\usepackage[
	textsize=scriptsize,
	textwidth=3cm,
	%disable
	]{todonotes}
\newcommand{\todoregion}[2]{\hl{#1}\todo{#2}\ }
\renewcommand{\indexname}{Appendix 2: Index}
	
\title{Ambiguity in Humor: Detecting English Puns}
\author{Viktoriia Tsosenko \\
Computational Linguistics Department\\
Heinrich Heine University D\"usseldorf\\
\texttt{Viktoriia.Tsosenko@hhu.de}
		}

\begin{document}

\maketitle

%\tableofcontents
\begin{abstract}
Humor is a complex linguistic phenomenon, and puns are one of the good examples of how ambiguity creates amusement. Studying puns is valuable not only for linguistic analysis, but also for computational linguistics, where detecting humor automatically remains a challenging task. This paper emphasizes on the task of homographic pun detection. For this purpose, a publicly available dataset the SemEval-2017 Task 7 is used. The aim of this paper is to explore ambiguity in homographic puns; to apply NLP techniques to detect puns automatically and to get experience with data handling and machine learning. 
\end{abstract}


\section{Introduction}

Human communication is difficult to imagine without humor. “If computers are ever going to communicate naturally and effectively with humans, they must be able to use humor.” (Binsted, 2006, p.59). Humor also has its “beneficial aspects, because it banishes sadness and boredom, puts the individual in an optimistic mood and, in general, lightens the load of everyday living.” (Larkin-Galiñanes, 2017, p.4). Making machines “understand” and “produce” humor is therefore beneficial for people. Among the various forms of humor, puns are particularly interesting because they exploit the multiple meanings of words or the similarity of sounds to generate unexpected interpretations. It is known that words often have multiple meanings, and Zipf (1949) indicates that the most frequently used words tend to have more senses than less frequent ones. 


\section{Humor and Ambiguity}

The perception of what humor is has undergone major changes throughout the history. Larkin-Galiñanes (2017) believes that humor as we know it today, originates from the 20th century. There is no single, universally accepted definition of humor though. Some writers believe it is even impossible to define what exactly humor, as a term, means (Attardo, 1994).
Plenty of linguistic devices are used to create humor. In Ancient Greece and Rome, Aristotle and other classic writers tried to find out what linguistic techniques were used to produce laughter, and ambiguity was listed among them. In Rhetorics, Aristotle talks about “twisting from the proper and apparent sense”, or giving a word “a different ‘turn’” that awakes laughter. He mentions “temporary deception practiced upon the listener”, comparing it with some kind of riddle. When listening to a joke, the listener “does not at once trace the resemblance”, but by “discovering its meaning, the listener learns something new” and is therefore experiencing a satisfying mental “aha!” moment. (Aristotle, Translation, 2009, p.319-320). This is a good evidence that ambiguity in humor has interested people starting from Ancient Times.  

The examples below illustrate how ambiguity works at different linguistic levels (morphological, lexical, syntactic etc.), and how humor depends on it.

\paragraph{Structural ambiguity} Dubinsky presents a joke from the Dilbert cartoon (Dubinsky, 2011, p.56): 

\begin{quote}
\textbf{Person A to Person B:} Would you like to buy advertising in my new magazine called 
\textit{Gullible World}? We have between one and two billion readers!\\
\textbf{Some time later, Person A to Person C:} I figured out how to make three readers sound like a lot.
\end{quote}


The humorous phrase \textit{one and two billion readers} could mean:
\begin{itemize}
    \item [[one and two] billion]
    \item [one] and [two billion]
\end{itemize}

\paragraph{Compound words} (Dubinsky, 2011, p.44): 


a)	the word man: “while a milkman delivers milk, you really don’t want your garbagemen or firemen bringing garbage or fire.”

b)	“In noun + noun compounds, sometimes the first noun tells you what the second one is made of or with, as in cheesecake or apple pie. Sometimes you hope that it doesn’t, as in shepherd pie (more commonly known as shepherd’s pie) or Girl-scout cookies. Or you find out that it does, and wish you hadn’t, as in headcheese.”

\paragraph{Idioms}  (Dubinsky, 2011, p.46), illustrated with an idiom “a piece of me” (meaning “challenge to fight”) and its literal meaning:

Frankenstein’s monster gets into an argument at a bar and says: “You want a piece of me? Huh? Huh? You want a piece of me?”




\section{Classification of Puns}\label{key}
“Punning is a form of humorous wordplay based on semantic ambiguity between two phonologically similar words – the pun and the target – in a context where both meanings are more or less acceptable.” (Palmann, 2025, p.1)

There are different types of puns. The most relevant ones for natural language processing application are homophonic and homographic puns.


\paragraph{Homophones} Homophonic puns sound the same, but their spelling is different. 
Dubinsky (2011, p.51) refers to a clear example from a 2001 FoxTrot panel, where Jason Fox and his friend Marcus show seven carved pumpkins in a row.  On the pumpkins, they carved, in order: the number “3,” a decimal point, then “1,” “4,” “1,” “5,” and “9.”Jason tells his sister: We’re calling it “pumpkin pi.”

The joke relies on the homophone pair pie (as in pumpkin pie) and pi (a mathematical constant). 

\paragraph{Homographs} Homographic puns share the same spelling. 

Example: I used to be a banker, but I lost interest.

Interest can mean curiosity and money earned on savings.

It is important to mention congruity between the two senses when talking about puns. “Nothing is more futile than the irrelevant pun that is based on only a verbal similarity and brings out no contrast, innuendo, or congruity of meaning” (Stanford, 1972, p.72). With no semantic opposition in the punning text, it will not be a joke anymore (Hempelmann et al., 2017). Raskin expressed that puns, based “on purely phonetical and not semantical relations between words” are called “bad puns” (1985, p.116). 

Understanding a pun often requires recognizing multiple meanings simultaneously, which is easy and intuitive for humans, but remains difficult for machines. Next chapters are dedicated to the automatic pun detection, namely of homographs. 

\section{Dataset}
The SemEval-2017 Task 7 includes six separate files with homographic puns and six with heterographic ones. For this paper, only the files with homographic puns were taken, split into 3 subsets, namely: a training set (70%), a validation set (15%) and a test set (15%). 
The dataset consists of three pairs of files: one .xml file with the sentences and one corresponding .gold file with annotations.
It is structured as follows:

Subtask 1: Pun Detection
The .xml file contains 2250 annotated sentences, each with its own sentence ID.
Each token in a sentence also has its own ID.
The corresponding .gold file provides labels for each sentence ID (1 for pun, 0 for no pun)
Subtask 2: Pun Location
The .xml file includes only pun sentences (1607)
The .gold file specifies the position of the punning word in each sentence with its ID.
Subtask 3: Sense Annotation
The .xml file contains 1298 sentences with puns, where the punning word is annotated with two different WordNet sense IDs, representing its two possible meanings in context.
The .gold file with these annotations to indicate the correct senses.


\end{document}

